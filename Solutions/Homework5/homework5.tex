\input{src/header}											% bindet Header ein (WICHTIG)
\usepackage{graphicx}
\usepackage{amsmath}
\usepackage{amssymb}

\newcommand{\dozent}{Prof. Dr. Margarita Esponda}					% <-- Names des Dozenten eintragen
\newcommand{\tutor}{Lilli Walter}						% <-- Name eurer Tutoriun eintragen
\newcommand{\tutoriumNo}{6}				% <-- Nummer im KVV nachschauen
\newcommand{\projectNo}{5}									% <-- Nummer des Übungszettels
\newcommand{\veranstaltung}{Nichtsequentielle Programmierung}	% <-- Name der Lehrveranstaltung eintragen
\newcommand{\semester}{SoeSe 2017}						% <-- z.B. SoSo 17, WiSe 17/18
\newcommand{\studenten}{Boyan Hristov, Sergelen Gongor}			% <-- Hier eure Namen eintragen
% /////////////////////// BEGIN DOKUMENT /////////////////////////


\begin{document}
\input{src/titlepage}										% erstellt die Titelseite


Link zum Git Repository: \url{https://github.com/BoyanH/FU-Berlin-ALP4/tree/master/Solutions/Homework5}

% /////////////////////// Aufgabe 1 /////////////////////////

\section*{Aufgabe 1}


\section*{Aufgabe 2}

\begin{enumerate}

\item[n=4, k=2]

Die erste zwei Threads laufen bis $p_8$ nach einander. D.h. $ count=0 \land D = 0$. Dan laufen zwei weitere Threads nach einander bis $p_7$. D.h. $count=-2$ und die letzte beide Threads warten auf D. Danach laufen die erste zwei Threads bis $p_1$. Das erste Thread setzt $count = count + 1 = -2 + 1 = =-1$ und released deswegen D. Dann wird das 2 Thread, das sich im CS befindet bis $p_1$ ausgeführt, setzt dabei $count = count+1 = -1 + 1 = 0 \leq 0 \Rightarrow$ released D wieder. Damit haben wir $D=2$ erreicht, ein unerlaubtes Zustand. 

\item[n=3, k=2]

\end{enumerate}

% /////////////////////// END DOKUMENT /////////////////////////
\end{document}